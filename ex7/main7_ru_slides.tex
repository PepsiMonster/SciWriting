\documentclass[handout,xcolor={svgnames}]{beamer} 

\usetheme{Warsaw}
\usecolortheme{beaver}

\usepackage[utf8]{inputenc}
\usepackage[russian]{babel}
\usepackage{amsmath,amssymb}
\usepackage{graphicx}
\usepackage{tikz}

\title{5G сети и распространение сигнала}
\author{Лобов Михаил}
\institute{RUDN University}
\date{\today}

\begin{document}

\begin{frame}
    \titlepage
\end{frame}

\begin{frame}{Что такое 5G?}
\begin{block}{Кратко}
    \begin{itemize}
        \item<1-> 5G --- это мобильные сети нового поколения.
        \item<2-> Используют более высокие частоты (например, около \(3{,}5\,\text{ГГц}\) и выше).
        \item<3-> Цели: более высокая скорость, меньшие задержки,
        поддержка большого числа устройств (IoT).
\end{itemize}
\end{block}

\pause

\begin{block}{Роль базовых станций}
    \begin{itemize}
        \item<4-> Соты стали меньше по размеру, но их стало больше.
        \item<5-> Для покрытия города нужно плотное размещение станций.
    \end{itemize}
\end{block}
\end{frame}

\begin{frame}{Распространение сигнала в 5G}
  \begin{block}{Основная идея}
    Радиосигнал от вышки до пользователя ослабляется по пути:
    \[
      P_{\text{получ}} \propto \frac{1}{d^{\alpha}},
    \]
    где \(d\) — расстояние до вышки, а \(\alpha \approx 2\text{--}4\) —
    коэффициент затухания.
  \end{block}

  \pause

  \begin{block}{Особенности высоких частот}
    \begin{itemize}
      \item<2-> Сильнее поглощение стенами и препятствиями.
      \item<3-> Сложнее покрыть большие площади одной вышкой.
      \item<4-> Важна \alert{прямая видимость} (line-of-sight).
    \end{itemize}
  \end{block}
\end{frame}

\begin{frame}{Шум и отношение сигнал/шум}
    \begin{block}{Шум}
        \begin{itemize}
            \item<1-> Тепловой шум приемника.
            \item<2-> Помехи от электроники и других устройств.
        \end{itemize}
    \end{block}

\pause

\begin{block}{Signal to noise ratio}
    \uncover<3->{Качество связи часто описывают величиной}
    \[
      \uncover<3->{\text{SNR}
        = \frac{P_{\text{сигнала}}}{P_{\text{шума}}}.}
    \]

    \uncover<4->{Чем больше \(\text{SNR}\), тем надёжнее можно передавать данные.}
    \uncover<5->{При маленьком \(\text{SNR}\) растёт вероятность ошибок,
      приходится снижать скорость.}
\end{block}
\end{frame}

\begin{frame}{Интерференция}
    \frametitle{Интерференция в сотовой сети}

    \begin{columns}
        \begin{column}{0.55 \textwidth}
            \begin{block}{Источники интерференции}
                \begin{itemize}
                    \item<1-> Соседние соты, работающие на той же или близкой частоте.
                    \item<2-> Другие пользователи внутри одной соты.
                    \item<3-> Отражения сигнала от зданий (\alert{многолучевость})
                \end{itemize}
            \end{block}

\pause

\begin{block}{Эффект}
    \uncover<4-> {Сумма сигналов может как усиливать, так и ослаблять полезный сигнал.}
    \uncover<5-> {Это приводит к провалам уровня сигнала или ошибкам при приёме.}
\end{block}
\end{column}

\begin{column}{0.45\textwidth}
        \begin{center}
        \uncover<2->{
          \begin{tikzpicture}[scale=0.9]
            \draw (0,0) node[circle,draw] {} node[above]{Вышка 5G};
            \draw[->] (0,0) -- (2,1) node[right]{Пользователь A};
            \draw[->] (0,0) -- (2,-1) node[right]{Пользователь B};
          \end{tikzpicture}
        }
      \end{center}
    \end{column}
\end{columns}
\end{frame}

\begin{frame}{Методы борьбы с шумом и помехами}
    \begin{enumerate}
        \item \uncover<1-> {\alert{Множественный доступ и планирование частот}
        \\
        Аккуратное распределение частот между сотами и пользователями.}
        \item \uncover<2-> {\alert{MIMO \& Beamforming}
        \\
        Использование нескольких антенн для формирования узких лучей в сторону конкретного пользователя.}
        \item \uncover<3-> {\alert{Малые соты}\\
        Увеличение количества базовых станций, меньше расстояния, лучше \(\text{SNR}\)
        --- более устойчивые схемы.}
    \end{enumerate}
\end{frame}

\begin{frame}{Итоги}
    \begin{block}{Основные идеи}
        \begin{itemize}
            \item<1-> 5G использует более высокие частоты, поэтому сигнал быстрее затухает с увеличением расстояния и чувствителен к препятствиям.
            \item<2-> Качество связи определяется уровнем шума и интерференции - \(\text{SNR}\).
            \item<3-> Для борьбы с проблемами применяются MIMO, beamforming, малые соты и умное управление ресурсами сети.
        \end{itemize}
    \end{block}
\end{frame}

\end{document}