\documentclass[a4paper,12pt]{article}

\usepackage[T2A]{fontenc}
\usepackage[utf8]{inputenc}
\usepackage[main=russian,english]{babel}
\usepackage{graphicx}
\usepackage{geometry}
\usepackage{hyperref}

\geometry{margin=2.5cm}

\title{Отчёт по установке \LaTeX{} и первой компиляции документа}
\author{Лобов Михаил Сергеевич}
\date{26 сентября 2025~г.}

\begin{document}

\maketitle

\section*{Цель работы}

Установить дистрибутив TeX Live под Windows~11 с помощью менеджера пакетов
Chocolatey и проверить работоспособность системы, выполнив компиляцию
простого \LaTeX-документа из командной строки.

\section*{Используемое программное обеспечение}

\begin{itemize}
  \item Операционная система: Windows~11.
  \item Менеджер пакетов: Chocolatey.
  \item Дистрибутив \LaTeX: TeX Live~2025
        (\url{https://www.tug.org/texlive/}).
  \item Редактор исходного текста: Visual Studio Code.
\end{itemize}

\section*{Ход работы}

\begin{enumerate}
  \item Был открыт PowerShell от имени администратора.
  \item Установка TeX Live выполнена командой
        \texttt{choco install texlive}.
  \item После завершения установки проверена доступность компилятора
        командой
        \texttt{pdflatex\ --version}. В выводе команды указано, что
        используется TeX Live~2025, что подтверждает корректную установку.
  \item В каталоге \texttt{C:\textbackslash Users\textbackslash kotof\textbackslash Study\textbackslash SciWriting}
        создан файл \texttt{main.tex} с минимальной структурой
        \LaTeX-документа (\texttt{\textbackslash documentclass},
        \texttt{\textbackslash begin\{document\}} и
        \texttt{\textbackslash end\{document\}}).
  \item В интегрированном терминале VS~Code запущена компиляция документа
        командой \texttt{pdflatex main.tex}.
  \item Из вывода компилятора видно, что без ошибок был создан файл
        \texttt{main.pdf} (см. рис.~\ref{fig:terminal}).
\end{enumerate}

При первом запуске компиляции пустого файла \texttt{test.tex} \LaTeX{}
перешёл в режим ожидания ввода команд и выдал сообщение о незавершённой
структуре документа. После добавления стандартного каркаса
документа компиляция стала проходить успешно.

\section*{Результаты}

В результате проделанной работы:

\begin{itemize}
  \item На рабочей станции установлен актуальный дистрибутив TeX Live~2025.
  \item Компилятор \texttt{pdflatex} доступен из PowerShell и терминала VS~Code.
  \item Успешно скомпилирован тестовый документ \texttt{main.tex} в файл
        \texttt{main.pdf}.
\end{itemize}

\section*{Выводы}

Цель работы достигнута: система \LaTeX{} корректно установлена и готова
к использованию для выполнения последующих заданий по научному письму
и вёрстке документов. Получен первый рабочий пример компиляции простого
документа из командной строки, что позволит в дальнейшем сосредоточиться
на содержательной части заданий (формулы, рисунки, библиография).

\begin{figure}[h]
  \centering
  \includegraphics[width=0.95\linewidth]{figure.png}
  \caption{Output of the \texttt{pdflatex main.tex} command in the terminal.}
  \label{fig:terminal-en}
\end{figure}

\end{document}
