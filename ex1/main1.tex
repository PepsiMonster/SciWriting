\documentclass[a4paper,12pt]{article}

\usepackage[T1]{fontenc}
\usepackage[utf8]{inputenc}
\usepackage[english]{babel}
\usepackage{graphicx}
\usepackage{geometry}
\usepackage{hyperref}

\geometry{margin=2.5cm}

\title{Report on \LaTeX{} installation and first document compilation}
\author{Lobov Mikhail Sergeevich}
\date{September 26, 2025}

\begin{document}

\maketitle

\section*{Goal}

The goal of this work is to install the TeX Live distribution on Windows~11
using the Chocolatey package manager and to verify that the system works
correctly by compiling a simple \LaTeX{} document from the command line.

\section*{Software}

\begin{itemize}
  \item Operating system: Windows~11.
  \item Package manager: Chocolatey.
  \item \LaTeX{} distribution: TeX Live~2025
        (\url{https://www.tug.org/texlive/}).
  \item Text editor: Visual Studio Code.
\end{itemize}

\section*{Procedure}

\begin{enumerate}
  \item PowerShell was started with administrator privileges.
  \item TeX Live was installed via Chocolatey using the command
        \texttt{choco install texlive}.
  \item After the installation finished, the availability of the compiler
        was checked with \texttt{pdflatex\ --version}. The output reports
        TeX Live~2025, which confirms a successful installation.
  \item A file \texttt{main.tex} with minimal \LaTeX{} structure
        (\texttt{\textbackslash documentclass},
        \texttt{\textbackslash begin\{document\}} and
        \texttt{\textbackslash end\{document\}}) was created in the directory
        \texttt{C:\textbackslash Users\textbackslash kotof\textbackslash Study\textbackslash SciWriting}.
  \item Compilation of the document was started from the integrated
        VS~Code terminal using the command \texttt{pdflatex main.tex}.
  \item According to the compiler output, the file \texttt{main.pdf} was
        generated without errors (see Fig.~\ref{fig:terminal-en}).
\end{enumerate}

During the first attempt to compile an almost empty file \texttt{test.tex},
\LaTeX{} switched to interactive mode and complained about an incomplete
document structure. After adding the standard document skeleton, the
compilation process finished successfully.

\section*{Results}

As a result of the work:

\begin{itemize}
  \item The current TeX Live~2025 distribution is installed on the machine.
  \item The \texttt{pdflatex} compiler is available from PowerShell and the
        VS~Code terminal.
  \item A test document \texttt{main.tex} was successfully compiled into
        \texttt{main.pdf}.
\end{itemize}

\section*{Conclusion}

The objective of the task has been achieved: \LaTeX{} is correctly installed
and ready to be used for further assignments in scientific writing and
typesetting. A first working example of document compilation from the
command line has been obtained, which allows focusing on the content
(formulas, figures, bibliography) in the next tasks.

\begin{figure}[h]
  \centering
  \includegraphics[width=0.95\linewidth]{figure.png}
  \caption{Output of the \texttt{pdflatex main.tex} command in the terminal.}
  \label{fig:terminal-en}
\end{figure}

\end{document}
