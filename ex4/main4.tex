\documentclass[a4paper,12pt]{article}
% For the \textwidth vs \linewidth experiment, compile once with this instead:
% \documentclass[a4paper,12pt,twocolumn]{article}

\usepackage[T1]{fontenc}
\usepackage{graphicx}
\usepackage{lipsum}
\usepackage{float}      % for [H]
\usepackage[hidelinks]{hyperref}

% Put your images in a subfolder (recommended):
% Create a folder named "figs" next to this .tex file.
\graphicspath{{figs/}}

\begin{document}

\section{Including graphics}
This document is a sandbox for Exercise 4.

\subsection{1) A centered image (basic)}
Below is a centered image. Replace \texttt{example-image} with your own file
(e.g.\ \texttt{myplot.pdf} or \texttt{photo.jpg}) placed in \texttt{figs/}.

\begin{center}
\includegraphics[height=2cm]{example-image}
\end{center}

\section{Altering the appearance of images}
\subsection{2) Height vs width}
Same image, two different ways to size it:

\begin{center}
\includegraphics[height=0.35\textheight]{example-image}
\end{center}

Some text in between.

\begin{center}
\includegraphics[width=0.55\textwidth]{example-image}
\end{center}

\subsection{3) Scale and rotation}
\begin{center}
\includegraphics[scale=0.6, angle=15]{example-image}
\end{center}

\subsection{4) Trimming and clipping}
The values in \texttt{trim} are \emph{left bottom right top}.
Try changing them and recompiling.

\begin{center}
\includegraphics[clip, trim=20 10 80 40, width=0.6\textwidth]{example-image}
\end{center}

\section{Making images float}
\subsection{5) Floats with different placement specifiers}
We add filler text so floats have a reason to move.

\lipsum[1-3]

This is the \emph{test location} in the source.

\begin{figure}[ht]
  \centering
  \includegraphics[width=0.6\textwidth]{example-image-a}
  \caption{A float with \texttt{[ht]} placement.}
  \label{fig:ht}
\end{figure}

\lipsum[4-6]

\begin{figure}[tb]
  \centering
  \includegraphics[width=0.6\textwidth]{example-image-b}
  \caption{A float with \texttt{[tb]} placement.}
  \label{fig:tb}
\end{figure}

\lipsum[7-9]

\begin{figure}[p]
  \centering
  \includegraphics[width=0.6\textwidth]{example-image-c}
  \caption{A float with \texttt{[p]} placement (float-only page).}
  \label{fig:p}
\end{figure}

\lipsum[10-11]

\subsection{6) Forcing exact placement with \texttt{[H]}}
The option \texttt{[H]} (from the \texttt{float} package) tries to place the figure
exactly “here”. It can create large whitespace, so it is not always recommended.

\begin{figure}[H]
  \centering
  \includegraphics[width=0.55\textwidth]{example-image}
  \caption{A float forced \texttt{[H]} (absolutely here).}
  \label{fig:H}
\end{figure}

\section{Cross-referencing}
\subsection{7) Referencing figures}
See Figure~\ref{fig:ht}, Figure~\ref{fig:tb}, Figure~\ref{fig:p}, and Figure~\ref{fig:H}.

If you see \textbf{??} the first time, compile again. Cross-references usually need
at least \(2\) runs.

\subsection{8) \texttt{\textbackslash textwidth} vs \texttt{\textbackslash linewidth}}
The difference is easiest to notice with \texttt{twocolumn}. Compile once without
\texttt{twocolumn}, then enable it in the \texttt{\textbackslash documentclass} line and compare.

\begin{figure}[ht]
  \centering
  \includegraphics[width=0.8\textwidth]{example-image}
  \caption{Width set relative to \texttt{\textbackslash textwidth}.}
  \label{fig:textwidth}
\end{figure}

\begin{figure}[ht]
  \centering
  \includegraphics[width=0.8\linewidth]{example-image}
  \caption{Width set relative to \texttt{\textbackslash linewidth}.}
  \label{fig:linewidth}
\end{figure}

\section{Label placement experiments}
\subsection{9) What if \texttt{\textbackslash label} is before \texttt{\textbackslash caption}?}
Correct placement is \emph{after} (or inside) \texttt{\textbackslash caption}.
Below we intentionally do it wrong to observe the result.

\begin{figure}[ht]
  \centering
  \includegraphics[width=0.55\textwidth]{example-image}
  \label{fig:wronglabel} 
  \caption{This figure has \texttt{\textbackslash label} before \texttt{\textbackslash caption} (wrong on purpose).}
\end{figure}

Now compare references:
Correct one: Figure~\ref{fig:linewidth}.  
Wrong one (watch what number it picks): Figure~\ref{fig:wronglabel}.

\subsection{10) Equation labels: inside vs after the environment}
\begin{equation}
e^{i\pi}+1 = 0
\label{eq:inside}
\end{equation}

Reference (correct): Equation~\ref{eq:inside}.

Now we label \emph{after} the equation environment (wrong on purpose):

\begin{equation}
a^2 + b^2 = c^2
\end{equation}
\label{eq:outside} 

Reference (watch what happens): Equation~\ref{eq:outside}.

\end{document}
