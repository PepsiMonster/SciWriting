\documentclass[a4paper,12pt]{article}
% Для эксперимента \textwidth vs \linewidth скомпилируй один раз так:
% \documentclass[a4paper,12pt,twocolumn]{article}

\usepackage[T2A]{fontenc}
\usepackage[utf8]{inputenc}
\usepackage[russian]{babel}

\usepackage{graphicx}
\usepackage{lipsum}
\usepackage{float}      % для [H]
\usepackage[hidelinks]{hyperref}

% Картинки в подпапке (рекомендуется):
% Создай папку "figs" рядом с этим .tex файлом.
\graphicspath{{figs/}}

\begin{document}

\section{Вставка изображений}
Этот документ — «песочница» для задания 4.

\subsection{1) Центрированное изображение (база)}
Ниже — центрированная картинка. Замени \texttt{example-image} на свой файл
(например, \texttt{myplot.pdf} или \texttt{photo.jpg}), положенный в \texttt{figs/}.

\begin{center}
\includegraphics[height=2cm]{example-image}
\end{center}

\section{Изменение вида изображений}
\subsection{2) Высота против ширины}
Одна и та же картинка, два способа задать размер:

\begin{center}
\includegraphics[height=0.35\textheight]{example-image}
\end{center}

Немного текста между примерами.

\begin{center}
\includegraphics[width=0.55\textwidth]{example-image}
\end{center}

\subsection{3) Масштаб и поворот}
\begin{center}
\includegraphics[scale=0.6, angle=15]{example-image}
\end{center}

\subsection{4) Обрезка и отсечение (trim + clip)}
Значения в \texttt{trim} задаются как \emph{left bottom right top}.
Попробуй менять числа и пересобирать PDF.

\begin{center}
\includegraphics[clip, trim=20 10 80 40, width=0.6\textwidth]{example-image}
\end{center}

\section{Плавающие изображения (floats)}
\subsection{5) Floats с разными спецификаторами размещения}
Добавим «рыбный» текст, чтобы у LaTeX был повод двигать объекты.

\lipsum[1-3]

Это \emph{тестовое место} (test location) в исходнике.

\begin{figure}[ht]
  \centering
  \includegraphics[width=0.6\textwidth]{example-image-a}
  \caption{Плавающее изображение со спецификатором \texttt{[ht]}.}
  \label{fig:ht}
\end{figure}

\lipsum[4-6]

\begin{figure}[tb]
  \centering
  \includegraphics[width=0.6\textwidth]{example-image-b}
  \caption{Плавающее изображение со спецификатором \texttt{[tb]}.}
  \label{fig:tb}
\end{figure}

\lipsum[7-9]

\begin{figure}[p]
  \centering
  \includegraphics[width=0.6\textwidth]{example-image-c}
  \caption{Плавающее изображение со спецификатором \texttt{[p]} (страница только для float).}
  \label{fig:p}
\end{figure}

\lipsum[10-11]

\subsection{6) Принудительное размещение \texttt{[H]}}
Опция \texttt{[H]} (из пакета \texttt{float}) пытается поставить рисунок
строго «здесь». Это может создать большие пустые области, поэтому использовать
\texttt{[H]} стоит осторожно.

\begin{figure}[H]
  \centering
  \includegraphics[width=0.55\textwidth]{example-image}
  \caption{Float с \texttt{[H]} (строго здесь).}
  \label{fig:H}
\end{figure}

\section{Перекрёстные ссылки}
\subsection{7) Ссылки на рисунки}
Смотри рисунок~\ref{fig:ht}, рисунок~\ref{fig:tb}, рисунок~\ref{fig:p} и рисунок~\ref{fig:H}.

Если при первом прогоне видишь \textbf{??}, просто скомпилируй ещё раз:
перекрёстные ссылки обычно требуют минимум \(2\) прогона.

\subsection{8) \texttt{\textbackslash textwidth} против \texttt{\textbackslash linewidth}}
Разницу проще всего увидеть с \texttt{twocolumn}. Скомпилируй один раз без
\texttt{twocolumn}, затем включи его в строке \texttt{\textbackslash documentclass} и сравни.

\begin{figure}[ht]
  \centering
  \includegraphics[width=0.8\textwidth]{example-image}
  \caption{Ширина задана относительно \texttt{\textbackslash textwidth}.}
  \label{fig:textwidth}
\end{figure}

\begin{figure}[ht]
  \centering
  \includegraphics[width=0.8\linewidth]{example-image}
  \caption{Ширина задана относительно \texttt{\textbackslash linewidth}.}
  \label{fig:linewidth}
\end{figure}

\section{Эксперименты с размещением \texttt{\textbackslash label}}
\subsection{9) Что будет, если \texttt{\textbackslash label} стоит до \texttt{\textbackslash caption}?}
Правильно ставить \texttt{\textbackslash label} \emph{после} (или внутри) \texttt{\textbackslash caption}.
Ниже мы намеренно делаем неправильно, чтобы увидеть результат.

\begin{figure}[ht]
  \centering
  \includegraphics[width=0.55\textwidth]{example-image}
  \label{fig:wronglabel} % НАРОЧНО неправильно: до \caption
  \caption{Здесь \texttt{\textbackslash label} стоит перед \texttt{\textbackslash caption} (ошибка специально).}
\end{figure}

Теперь сравни ссылки:
Правильная: рисунок~\ref{fig:linewidth}. \\
Неправильная (посмотри, какой номер подставится): рисунок~\ref{fig:wronglabel}.

\subsection{10) Метка у формулы: внутри окружения и после него}
\begin{equation}
e^{i\pi}+1 = 0
\label{eq:inside}
\end{equation}

Ссылка (правильно): формула~\ref{eq:inside}.

А теперь ставим метку \emph{после} окружения (ошибка специально):

\begin{equation}
a^2 + b^2 = c^2
\end{equation}
\label{eq:outside} 

Ссылка (посмотри, что получится): формула~\ref{eq:outside}.

\end{document}
