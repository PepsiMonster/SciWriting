\documentclass{article}
\usepackage[T1]{fontenc}
\usepackage[utf8]{inputenc} 
\usepackage[russian]{babel}
\usepackage{array}
\usepackage{booktabs}

\begin{document}

\section*{Упражнение 5.14 --- Таблицы (пакет \texttt{array})}

\subsection*{1) Начнём с простого примера таблицы}

\begin{tabular}{lll}
Животное & Еда  & Размер \\
собака   & мясо & средний \\
лошадь   & сено & большой \\
лягушка  & мухи & маленький \\
\end{tabular}

\bigskip

\subsection*{2) Попробуем разные выравнивания с помощью \texttt{l}, \texttt{c} и \texttt{r}}

\noindent Лево/Лево/Лево:
\begin{tabular}{lll}
Животное & Еда  & Размер \\
собака   & мясо & средний \\
лошадь   & сено & большой \\
лягушка  & мухи & маленький \\
\end{tabular}

\medskip
\noindent Центр/Центр/Центр:
\begin{tabular}{ccc}
Животное & Еда  & Размер \\
собака   & мясо & средний \\
лошадь   & сено & большой \\
лягушка  & мухи & маленький \\
\end{tabular}

\medskip
\noindent Право/Право/Право:
\begin{tabular}{rrr}
Животное & Еда  & Размер \\
собака   & мясо & средний \\
лошадь   & сено & большой \\
лягушка  & мухи & маленький \\
\end{tabular}

\bigskip

\subsection*{3) Что будет, если в строке слишком мало элементов?}

\noindent Если строка содержит меньше ячеек, чем объявлено в преамбуле (здесь \(3\) столбца),
то недостающие ячейки остаются пустыми.

\medskip
\begin{tabular}{lll}
Животное & Еда  & Размер \\
собака   & мясо & средний \\
лошадь   & сено \\        
лягушка  & мухи & маленький \\
\end{tabular}

\bigskip

\subsection*{4) Что будет, если в строке слишком много элементов?}

\noindent Если строка содержит больше ячеек, чем объявлено, LaTeX выдаст ошибку (часто
\emph{``Extra alignment tab has been changed to \textbackslash cr''}). Следующий код
\textbf{не} скомпилируется, если вставить его как есть:

\medskip
\begin{verbatim}
\begin{tabular}{lll}
Животное & Еда  & Размер \\
собака   & мясо & средний & ЛИШНЕЕ \\ % 4 ячейки вместо 3
лошадь   & сено & большой \\
\end{tabular}
\end{verbatim}

\noindent Чтобы исправить, нужно либо убрать лишнюю ячейку, либо объявить \(4\) столбца в преамбуле.

\bigskip

\subsection*{5) Эксперименты с \texttt{\textbackslash multicolumn} (объединение ячеек по горизонтали)}

\subsubsection*{5.1) Объединение \(2\) ячеек в одну}

\begin{tabular}{lll}
\toprule
Животное & Еда  & Размер \\
\midrule
собака   & мясо & средний \\
лягушка  & мухи & маленький \\
фуат     & \multicolumn{2}{c}{неизвестно} \\
\bottomrule
\end{tabular}

\bigskip

\subsubsection*{5.2) Отмена оформления для одной ячейки с помощью \texttt{\textbackslash multicolumn\{1\}\{...\}\{...\}}}

\noindent Здесь первый столбец набирается курсивом и после значения добавляется двоеточие,
но заголовок первого столбца оставляем обычным, используя \texttt{\textbackslash multicolumn}.

\medskip
\begin{tabular}{>{\itshape}l<{:}ll}
\toprule
\multicolumn{1}{l}{Животное} & Еда & Размер \\
\midrule
собака   & мясо & средний \\
лошадь   & сено & большой \\
лягушка  & мухи & маленький \\
\bottomrule
\end{tabular}

\end{document}
