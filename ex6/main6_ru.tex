\documentclass{article}
\usepackage[T1]{fontenc}
\usepackage[utf8]{inputenc} % для pdfLaTeX; при LuaLaTeX можно убрать
\usepackage[russian]{babel}
\usepackage{natbib}

% Создаём .bib-файл изнутри .tex (для автономности)
\begin{filecontents*}{learnlatex.bib}
@article{Thomas2008,
  author  = {Thomas, Christine M. and Liu, Tianbiao and Hall, Michael B. and Darensbourg, Marcetta Y.},
  title   = {Series of Mixed Valent {Fe(II)Fe(I)} Complexes That Model the {H(OX)} State of [{FeFe}]Hydrogenase:
             Redox Properties, Density-Functional Theory Investigation, and Reactivity with Extrinsic {CO}},
  journal = {Inorg. Chem.},
  year    = {2008},
  volume  = {47},
  number  = {15},
  pages   = {7009-7024},
  doi     = {10.1021/ic800654a},
}

@book{Graham1995,
  author    = {Ronald L. Graham and Donald E. Knuth and Oren Patashnik},
  title     = {Concrete Mathematics},
  publisher = {Addison-Wesley},
  year      = {1995},
}

% Новая запись для упражнения (создать новую запись в базе)
@book{Lamport1994,
  author    = {Leslie Lamport},
  title     = {LaTeX: A Document Preparation System},
  publisher = {Addison-Wesley},
  year      = {1994},
  edition   = {2},
}
\end{filecontents*}

\begin{document}

\section*{Упражнение 6.9 --- Библиография с \texttt{natbib} (рабочий процесс BibTeX)}

\subsection*{1) Базовые цитирования (текстовые и в скобках)}
Пример по математике взят из \citet{Graham1995}, а пример по химии --- из \citet{Thomas2008}.

Примеры ссылок в скобках: \citep{Graham1995} и затем \citep[p.~56]{Thomas2008}.

Ссылка с коротким примечанием и диапазоном страниц: \citep[См.][pp.~45--48]{Graham1995}.

Несколько источников в одной ссылке: \citep{Graham1995,Thomas2008}.

\subsection*{2) Новая запись и новая ссылка}
Вот ссылка на новую запись, которую мы добавили в базу: \citet{Lamport1994}.

\subsection*{3) Ссылка на источник, которого нет в базе}
Этого ключа нет в \texttt{learnlatex.bib}: \citep{NotInDatabase}.
(Должно появиться предупреждение, а в документе ссылка обычно отображается как \texttt{?}.)

\bibliographystyle{plainnat}
\bibliography{learnlatex}

\end{document}
