% !TEX program = lualatex
\documentclass[a4paper,12pt]{article}

\usepackage{fontspec}
\usepackage[russian]{babel}

% Любой системный шрифт с кириллицей:
\setmainfont{Times New Roman}

\usepackage{amsmath}
\usepackage{amsfonts}
\usepackage{bm}
\usepackage{mathtools}

\newcommand{\diff}{\mathop{}\!\mathrm{d}}

\begin{document}

\section*{Упражнение: математический режим}

\subsection*{1) Inline-математика}
Встроенная формула: \( y = mx + c \).
Ещё одна: \( 5^{2} = 3^{2} + 4^{2} \).
Супер- и подстрочные индексы: \( a^{b} \) и \( a_{b} \), а также \( x_{n}^{2} \).

Обрати внимание: в математике пробелы игнорируются, поэтому \( a\;\;b \) и \( ab \) почти одинаковы,
а для «ручных» пробелов есть команды вроде \( \, \), \( \; \), \( \quad \).

\subsection*{2) Display-математика}
\[
y = mx + c
\]

\subsection*{3) Греческие буквы (нижний/верхний регистр)}
Например: \( \alpha, \beta, \gamma, \theta, \lambda, \pi, \omega \) и \( \Gamma, \Delta, \Theta, \Lambda, \Pi, \Omega \).

\subsection*{4) Интеграл + аккуратный \texttt{dx}}
\[
\int_{-\infty}^{+\infty} e^{-x^2} \diff x
\]

\subsection*{5) Нумерованная формула}
\begin{equation}
(x+y)(x-y)=x^{2}-y^{2}
\end{equation}

\subsection*{6) amsmath: выравнивание (align) и текст внутри математики}
\begin{align}
Q_{n,0} &= 1, \quad Q_{0,k} = [k=0]; \\
Q_{n,k} &= Q_{n-1,k}+Q_{n-1,k-1}+\binom{n}{k},
\quad \text{для } n,k>0.
\end{align}

\subsection*{7) Матрицы}
\[
\begin{pmatrix}
a & b & c \\
d & e & f
\end{pmatrix}
\quad
\begin{bmatrix}
1 & 2 \\
-5 & -6
\end{bmatrix}
\]

\subsection*{8) Шрифты в математике + вложенность}
\[
\text{``size''} \neq \mathit{size} \neq \mathrm{size} \neq \mathbf{size}
\]
\[
\mathrm{ABC}\quad
\mathit{ABC}\quad
\mathbf{ABC}\quad
\mathsf{ABC}\quad
\mathtt{ABC}\quad
\mathbb{R}\ \mathbb{N}\ \mathbb{Z}
\]
\[
\mathbf{\mathsf{A}} \quad \mathsf{\mathbf{A}} \quad \bm{\alpha} \quad \mathbf{\alpha}
\]

\subsection*{9) gather и multline}
\begin{gather}
P(x)=ax^{5}+bx^{4}+cx^{3}+dx^{2}+ex+f\\
x^2+x=10
\end{gather}

\begin{multline*}
(a+b+c+d)x^{5}+(b+c+d+e)x^{4} \\
+(c+d+e+f)x^{3}+(d+e+f+a)x^{2}+(e+f+a+b)x\\
+(f+a+b+c)
\end{multline*}

\end{document}
