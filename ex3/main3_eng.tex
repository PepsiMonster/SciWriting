\documentclass [a4paper,12pt]{article}

\usepackage[T1]{fontenc}
\usepackage{amsmath}
\usepackage{amssymb}      % Добавлено для \mathbb
\usepackage{bm}
\usepackage{mathtools}

\newcommand{\diff}{\mathop{}\!\mathrm{d}}  % Исправлено

\begin{document}

\section*{Exercise: Math mode}

\subsection*{1) Inline math}
Inline formula: \(y= mx+c\).
Another example: \(5^{2}= 3^{2} + 4^{1000}\).
Superscripts and subscripts: \(a^{b}\) or \(a_{b}\) and \(x_{n}^{1000}\).

Spaces does not register in this mode. For manual spacing use \( \, \), \( \; \), \( \quad \).

\subsection*{2) Display math}

The same formula can be showcased in the middle of the page on its own line:
\[
y = kx + c 
\]

Text will go on after this notation as normal. 

\subsection*{3) Greek letters}
For example: \( \alpha, \beta, \gamma, \theta, \lambda, \pi, \omega \) and \( \Gamma, \Delta, \Theta, \Lambda, \Pi, \Omega \).

\subsection*{4) Integral + neatly typeset \texttt{dx}}
\[
\int_{-\infty}^{+\infty} e^{-x^2} \diff x
\]

\subsection*{5) Numbered equations}
\begin{equation}
    (x+y)(x-y)=x^{2}-y^{2}
\end{equation}

\subsection*{6) amsmath: alignment and text inside math}
\begin{align}
Q_{n,0} &= 1, \quad Q_{0,k} = [k=0]; \\
Q_{n,k} &= Q_{n-1,k}+Q_{n-1,k-1}+\binom{n}{k},
\quad \text{for } n,k>0.
\end{align}

\subsection*{7) Matrices}  % Исправлено: была опечатка в слове Matricies
\[
\begin{pmatrix}
    a & b & c \\
    d & e & f
\end{pmatrix}
\quad
\begin{bmatrix}
    1 & 2 \\
    -5 & -6
\end{bmatrix}
\]

\subsection*{8) Fonts in math mode + nesting}

Compare:
\[
\text{``size''} \neq \mathit{size} \neq \mathrm{size} \neq \mathbf{size}
\]
Individual styles:
\[
\mathrm{ABC}\quad
\mathit{ABC}\quad
\mathbf{ABC}\quad
\mathsf{ABC}\quad
\mathtt{ABC}\quad
\mathbb{R}\ \mathbb{N}\ \mathbb{Z}
\]

Now nesting will be present after the compilation is complete. 
\[
\mathbf{\mathsf{A}} \quad \mathsf{\mathbf{A}} \quad \bm{\alpha} \quad \mathbf{\alpha}
\]

\subsection*{9) Two more environments: gather and multline}
\begin{gather}
P(x)=ax^{5}+bx^{4}+cx^{3}+dx^{2}+ex+f\\
x^2+x=10
\end{gather}
\begin{multline*}
(a+b+c+d)x^{5}+(b+c+d+e)x^{4} \\
+(c+d+e+f)x^{3}+(d+e+f+a)x^{2}+(e+f+a+b)x\\
+(f+a+b+c)
\end{multline*}

\end{document}