\documentclass[a4paper,12pt]{article}
\usepackage[T2A]{fontenc}
\usepackage[utf8]{inputenc}
\usepackage[russian]{babel}


\begin{document}

Привет, мир!

Это мой первый настоящий документ. Вот здесь много пробелов --- но в PDF они превратятся в одиночные пробелы.

Это новый абзац. Обратите внимание: пустая строка выше начинает новый абзац.

Теперь проверим неразрывный пробел (hard space) с помощью \texttt{\string~}:
Неразрывный пробел действительно работает.

Давайте создадим потенциальные переносы строк. Этот абзац намеренно сделан длинным, чтобы LaTeX мог решать, где переносить строки в разных местах в зависимости от ширины страницы и слов, которые появляются ближе к концу каждой строки.

\bigskip
Теперь выведем специальные символы (обычно они имеют особый смысл в LaTeX):

\begin{itemize}
  \item Фигурные скобки: \{ и \}
  \item Знак доллара: \$ (например, чтобы написать \$10)
  \item Процент: \% (иначе начинается комментарий)
  \item Амперсанд: \& (иначе используется в таблицах)
  \item Решётка: \# 
  \item Подчёркивание: \_ (в тексте нужно экранировать)
  \item Обратный слэш: \textbackslash
  \item «Крышечка» (caret): \textasciicircum
  \item Тильда: \textasciitilde
\end{itemize}

И добавим сноску\footnote{Это текст сноски: проверяем, что команды в документе работают.} для проверки.

\end{document}