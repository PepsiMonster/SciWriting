\documentclass[a4paper,12pt]{article}
\usepackage[T2A]{fontenc}
\usepackage[utf8]{inputenc}
\usepackage[russian]{babel}


\begin{document}

Привет, мир!

Это мой первый документ.
Вот   здесь     много        пробелов --- но в PDF они превратятся в одиночные пробелы.

Это новый абзац. Обрати внимание: пустая строка выше начинает новый абзац.

Теперь проверим неразрывный пробел (hard space) с помощью \texttt{\string~}:
Г-н~Иванов не должен переноситься на следующую строку после «Г-н».
Также это удобно для ссылок вроде стр.~10 или рис.~2 (позже пригодится).

Сделаем абзац подлиннее, чтобы LaTeX мог переносить строки в разных местах:
Этот абзац специально сделан длинным, чтобы LaTeX мог выбирать места переноса строк в зависимости от ширины страницы и расположения слов ближе к концу строки.

\bigskip
Теперь выведем специальные символы (обычно они имеют особый смысл в LaTeX):

\begin{itemize}
  \item Фигурные скобки: \{ и \}
  \item Знак доллара: \$ (например, чтобы написать \$10)
  \item Процент: \% (иначе начинается комментарий)
  \item Амперсанд: \& (иначе используется в таблицах)
  \item Решётка: \#
  \item Подчёркивание: \_ (в тексте нужно экранировать)
  \item Обратный слэш: \textbackslash
  \item «Крышечка» (caret): \textasciicircum
  \item Тильда: \textasciitilde
\end{itemize}

И добавим сноску\footnote{Текст сноски: проверяем, что команды в документе работают.} для проверки.

\end{document}
