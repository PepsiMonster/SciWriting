\documentclass[a4paper,12pt]{article}
\usepackage[T1]{fontenc}

\begin{document}

Hey world!

This is a first document.
Here   are     some      extra spaces --- but in the PDF they collapse to single spaces.

This is a new paragraph. Notice the blank line above: it starts a new paragraph.

Now we try a non-breaking space (hard space) using \texttt{\string~}:
Mr.~Smith cannot break across lines.
Also useful for things like p.~10 or fig.~2 later.

Let's force some potential line breaks by using a longer sentence:
This paragraph is intentionally long so LaTeX may decide to break lines in different places depending on page width and the words that appear near the end of each line.

\bigskip
Now let's print special characters (they normally have special meaning in LaTeX):

\begin{itemize}
  \item Curly braces: \{ and \}
  \item Dollar sign: \$ (so we can show money like \$10)
  \item Percent sign: \% (otherwise it starts a comment)
  \item Ampersand: \& (otherwise used in tables)
  \item Hash: \# 
  \item Underscore: \_ (often used in math, so needs escaping in text)
  \item Backslash: \textbackslash
  \item Caret: \textasciicircum
  \item Tilde: \textasciitilde
\end{itemize}

This sentence has a footnote\footnote{This is the footnote text.} to confirm commands work in the body.

\end{document}
